\section{Introduction}
\label{sec:intro}

The current trends in domain specific software engineering, domain specific
languages (DSL) and domain specific modelling languages (DSML) demonstrate the
interest for \emph{tailored} software solutions.
When it comes to model-driven engineering (MDE) tools, studies like \cite{DBLP:conf/models/WhittleHRBH13} 
show that this trend is justified: among the MDE tools considered
in the reported study, the ones which were successful in penetrating industry
were precisely those which were developed in a tailored manner for a given
audience, the recommendation being: ``Match tools to people, not the other way
around''.

Many technologies now precisely enable this sort of tailoring, among which 
JetBrains' MPS \cite{DBLP:conf/pppj/PechSV13},
Xtext \cite{DBLP:conf/oopsla/EysholdtB10},
Sirius \cite{DBLP:conf/asplos/HauswaldLZLRKDM15}, or the classic MetaEdit+ \cite{DBLP:conf/sle/Tolvanen16}.
With the maturity of these technologies, one can safely say that building your own MDE tool
has never been so easy.
This is an essential enabler and we can now clearly observe how technology
enthusiasts in various industries put this opportunity to good use, developping their own domain- or even company-specific tools.

On the other hand, as \cite{DBLP:conf/models/WhittleHRBH13} also mentions, tools are an enabler,
but they are not everything: ``More focus on processes, less on tools".
Even when a tailored tool is available, allowing the modelling of one's domain through
many sub-DSLs makes it such that new users often overwhelmed by the amount
of modelling techniques at their disposal. This is in fact the case for even
very specialized tools like Sfit \cite{DBLP:conf/vamos/BayhaLAMI16} for the
modelling of industrial manufacturing -- while modelling a restricted domain,
the tool contains a large number of different models, mostly rendered as
diagrams.
The question of methodology or process then naturally follows: in which order
should one use the diagrams? More generally, which information should one model
at a given point in time? The funding of several research projects precisely
focusing on this question, in particular in connection with MDE, demonstrate the relevance of this
question: for example SPES-XT \cite{DBLP:books/sp/spes2016},
targets the development of methodologies for the domain of embedded systems.
Similarly, Arcadia \cite{DBLP:conf/syscon/BonnetVEN16} emphasizes the importance
of the methodology in connection with MDE.

Just like for tools however, methodologies and processes can seldom be general
enough to match all use cases and answer all needs. There is thus also a need
for tailoring at that level. To facilitate the acceptance of the methodology, it
is essential that the tool \emph{supports} the methodology: this is for instance
the case with Capella and Arcadia.
Capella however, supports only the Arcadia methodology which is specific to
avionic systems engineering and to the processes of Thales.
All the above points to the fact that, if the tools can be tailored, the process
itself should be tailorable.

In this paper, we propose precisely an approach to develop DS(M)Ls in JetBrains'
MPS, \emph{equipped with a means to customize the tool in order to support a
given process:} using out approach, in addition to the usual MPS mechanisms to
develop their DSLs, developers can also explicitly express their own process.
Such a process of model construction is expressed declaratively as a
statechart-like diagram, being that the current state is defined by the
satisfaction of some properties of the model as well as by its previous states.
For each state, the tool developer can define hints to be displayed as well as
quickfixes in the form of creation of new artifacts.

In order to clarify our work we draw an analogy with the M-levels defined by the Object Management Group:
\begin{itemize}
  \item M3: the MPS tool with its language definition capabilities.
  \item M2: development and composition of ``brick DSLs'' in a
  domain-specific model-driven development environment, together with a
  model construction process.
  \item M1: usage of the developed domain-specific tool.
\end{itemize}
We also identify four different \emph{roles} in the development and
usage of the framework:
\begin{itemize}
  \item the \emph{framework developer} (in our case JetBrains), who develop MPS
  (level M3),
  \item the \emph{framework customizer} (the authors of this paper), who develop
  a library for process-customizable DSLs (level M2),
  \item the \emph{(domain-specific) tool developer} 
    (typically a consultant or the in-house technology department of a company)
    who actually develops the domain-specific tool, making use of our libraries (level M2),
  \item and the \emph{user} (level M1).
\end{itemize}

In this work, we contribute a framework at level M2 to support the tool developer
in developing a \emph{process-aware} domain-specific tool.
In particular we have implemented a so-called \textsf{Process} language for
describing a set of refinement steps including descriptions which are then used to guide and help the user.
At the M1 level, a ``dashboard'' allows the user to know permanently the next step to achieve.

As a case study, we demonstrate how to use this framework for the
specific domain of requirements engineering: the step-by-step formalization of requirements is a common approach,
making it, therefore, an ideal candidate for the development of a process-aware tool.
For this purpose, we have implemented a set of DSLs supporting the MIRA
\cite{MIRA13} framework, a general approach for the stepwise formalization of requirements
focusing on quality assurance for requirements.

A \emph{tool developer} can use this set of DSLs as a basis for their
model-based development environment, by composing them with more specific DSLs
of their own design.
They can also ``drive'' the user in their requirements formalization process by
implementing a dedicated process using our \textsf{Process} language.
We illustrate this approach by developing a requirements-engineering tool
specialized in the development of hardware cooling systems, inspired from a case
study coming from our industrial partners at Diehl Aerospace.

% The remainder of this paper is structured as follows.  In
% section~\ref{sec:metameta} we describe the MPS framework, which we use as the
% technological basis for all the results presented in this paper.
% Section~\ref{sec:meta} then describes our case study -- a set of languages and a
% process for the incremental gathering and refinement of requirements,
% specialized for hardware cooling systems. Then, in section~\ref{sec:model},
% we exemplify the construction of the requirements for a specific fan-based cooling system, using
% the previously defined languages and refinement process.
% Section~\ref{sec:implementation} lifts the veil over some of the implementation
% details of our work and section~\ref{sec:related_work} provides pointers to work
% in the literature that closely relates to the results we present here. Finally,
% section~\ref{sec:conclusion} presents a discussion of this research and 
% potential future work.

The remainder of this paper is structured as follows.  In
section~\ref{sec:metameta} we describe the MPS framework, which we use as the
technological basis for all the results presented in this paper.
Section~\ref{sec:meta} then describes our case study -- the construction of a
model-driven development environment for the incremental gathering and
refinement of requirements. Then, in section~\ref{sec:model}, we exemplify the
construction of the requirements using the environment described in the previous
section. Section~\ref{sec:related_work} provides pointers to work in the
literature that closely relates to the results we present here. Finally,
section~\ref{sec:conclusion} presents a discussion of our research and potential
future work.

