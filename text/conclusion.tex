\section{Conclusions and Future Work}
\label{sec:conclusion}
 \vspace{-.4cm}
We have presented a technique for the construction of
domain-specific model editors in MPS, where these editors are based on
a set of composed DSLs and on a description of the process that
should be followed when building the models for that domain. We have applied our
approach to the construction of an editor for gathering software requirements
at two levels: firstly, we build an abstract requirements gathering framework,
following the guidelines in the MIRA framework~\cite{MIRA13}; secondly, we
specialize that framework for our industrial partners at Diehl, by introducing
a specific requirements refinement process for controllers for cooling systems.

The main technical contribution of this paper are the means to define a process
to assist in building a model in a domain-specific model-driven development
environment. This process is based on a set of automated model analyses and can
guide the user until the model is complete.
At a methodological level, our contribution regards our ideas on the separation
of the \emph{framework customizer} and the  \emph{domain specific tool
developer} roles. While the former is responsible for defining a number of
fundamental ``brick'' DSLs, the latter further specializes them for a particular
application by adding more languages and organizing the whole according to a
process.

In its current state, the main shortcoming of the technique we propose in this
paper is the fact that it is the complete responsibility of the domain-specific
tool developer to check the consistency of the properties being checked during
the unfolding of the process, as well as their logical sequence in the process.
Additional assistance during this step could be envisaged, for example in the
form of automated checks for logical inconsistencies in the conditions that
define each state in the process. Scalability is also an issue as analyses
currently run very often in the background, which will rapidly lead to
performance degradation in larger systems. Potential solutions to this problem
are currently being investigated by colleagues of our at
fortiss~\cite{Models17Sudeep}.

As future work, beyond mitigating the shortcomings above, we will continue
working with Diehl Aerospace to develop the case study we present in this paper
into a usable requirements gathering system. Other partner companies have shown
interest in applying our work domains other than requirements engineering, which
points to building or assembling different language stacks as well as different
processes.
 \vspace{-.6cm}
