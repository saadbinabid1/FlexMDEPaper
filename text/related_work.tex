\section{Related Work}
\label{sec:related_work}
 \vspace{-.3cm}
Mechanisms for providing some kind of guidance to the domain-specific
language user are present, to a smaller or larger extent, in all DSL definition
workbenches. In most cases, that guidance is provided in the form of
correctness-by-construction (e.g. only correct models that conform to a
metamodel can be built), or a-posteriori checks for conformance to certain
well-formedness rules. This is the case for example for DSL workbenches such as
Sirius~\cite{DBLP:conf/asplos/HauswaldLZLRKDM15},
Xtext~\cite{DBLP:conf/oopsla/EysholdtB10}, AutoFocus3~\cite{AF315},
MetaEdit+\cite{DBLP:conf/sle/Tolvanen16} or MPS~\cite{DBLP:conf/pppj/PechSV13}
itself. However, none of those tools is capable of natively providing the means
to explicitely define a model construction process or methodology that can
assist users when building instances of DSLs specified in those environments.

Model construction processes naturally depends on the domain the DSLs are aimed
at. It is thus not surprising that the explicit notion of process is more
present in modelling environments that are specifically aimed at certain domains
-- as previously mentioned, the Capella tool is a model-driven engineering
solution for systems and software architecture engineering which enforces the
Arcadia~\cite{DBLP:conf/syscon/BonnetVEN16} methodology, aimed at specific
domains such as transportation, avionics, space or radar; the Soley tool
suite~\cite{soley}, dedicated to model-based data extraction, processing and
visualization, includes workflows as first-class citizens.
Workflows are sequences of model transformations leading to analysis results
that can be played (semi-)automatically, as well as recorded by the tool from a
set of user actions.

Coming back to generic workbenches for language constructions, there exists a
large body of work in the area of model transformation chaining to orchestrate
the flow of models to achieve a modelling goal. For example, authors such as
Wagelaar~\cite{wagelaar2006blackbox}  or Kolovos~\cite{Kolovos2008} propose
mechanisms for automatically orchestrating model transformations such that
certain modelling goals are achieved. In this area, the study which is the
closest to the proposal in this paper is the FTG+PM
framework~\cite{DBLP:conf/sdl/LucioMDVJ13,MustafizDLV12}. The FTG+PM defines an
explicit process for the execution of model transformations.
Enacting that process means that certain model transformations are performed
automatically, while for others the user will have to input data at given
points. The differences with the work we present in this paper have to do with
the fact that our process is non-invasive, and is aimed at advising the user
rather than executing a pre-defined work flow. With our approach automated
actions are proposed to the user, who remains in complete control of the model
edition process at all times.

Note although mechanisms like the one we propose in this paper do not exist in
DSL workbenches such Sirius or AF3, it is possible to code them by using those
workbenches' APIs and making usage of already existing analysis structures, as
we have done for the work we present here. One of the differences that we are
aware of regarding the AF3 workbench in particular, is that the guidance
mechanism we present in this paper has been fully implemented using MPSs native
mechanisms. Buiding the same mechanism in AF3 requires on the other hand
extending the workbench's core framework using Java.
 \vspace{-.6cm}

